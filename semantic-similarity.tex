\documentclass{beamer}
\usepackage{booktabs}
\usepackage{pdfpages}
\usepackage{mathtools}
\usepackage{enumerate}
\usepackage{multirow,tabularx}
\usepackage{booktabs}
\usepackage{pdfpages}
\usepackage{proof}
\usepackage{cancel}
\usepackage{chronology}
\usepackage{graphicx}
\usepackage{ulem}
\usepackage{amsmath}
\usepackage{amssymb}
\usepackage{color}
\usepackage{animate}
\usepackage{xr}

\PassOptionsToPackage{usenames,dvipsnames,svgnames}{xcolor}  
\usepackage{tikz}
\usepackage{tkz-graph}


\usepackage{wasysym}
\usepackage{proof}
\usepackage{cancel}
\usepackage{chronology}
\usepackage{graphicx}
\usepackage{ulem}
\usepackage{amsmath}
\usepackage{amssymb}
\usepackage{color}
\usepackage{xcolor}
\usepackage{soul}
%\usepackage{pstricks}
\setbeamertemplate{navigation symbols}{}

\newcommand{\norm}[1]{\left\lVert#1\right\rVert}
\newcommand{\el}{$\mathcal{EL}^{++}$}
\renewcommand{\Re}{\mathbb{R}}
\newcommand{\BigO}[1]{\ensuremath{\operatorname{O}\bigl(#1\bigr)}}
\newcommand{\myul}[2][blue]{\sethlcolor{#1}\hl{#2}\setulcolor{black}}

\newcommand<>{\cunderline}[3]{\only<#1>{#3}\only<#2>{\underline{#3}}}
\newcommand<>{\cem}[3]{\only<#1>{#3}\only<#2>{\ul{#3}}}
\newcommand<>{\cgray}[3]{\only<#1>{#3}\only<#2>{\textcolor{gray}{#3}}}
\newcommand<>{\colorize}[4]{\only<#1>{#4}\only<#2>{\textcolor{#3}{#4}}}

\setbeamertemplate{navigation symbols}{\insertslidenavigationsymbol}
%\setbeamertemplate{navigation symbols}{}
% \addtobeamertemplate{navigation symbols}{}{%
%     \usebeamerfont{footline}%
%     \usebeamercolor[fg]{footline}%
%     \hspace{1em}%
%     \insertframenumber/\inserttotalframenumber
% }

\mode<presentation>
{
\usecolortheme{crane}
%\useoutertheme{split}

\expandafter\def\expandafter\insertshorttitle\expandafter{%
  \insertshorttitle\hfill%
  \insertframenumber\,/\,\inserttotalframenumber}

\usefonttheme[onlysmall]{structurebold}
}
\renewcommand{\em}{\itshape}
\usepackage{pifont}
\definecolor{purple}{rgb}{1,0,1}
\definecolor{dred}{rgb}{0.7,0,0}
\definecolor{myred}{rgb}{1,0,0}
\definecolor{dblue}{rgb}{0,0,0.7}
\definecolor{dgreen}{rgb}{0,0.5,0}
\definecolor{myyellow}{rgb}{1,1,0}
\newcommand{\parents}[1]{parents(#1)}
\setbeamertemplate{itemize item}[ball]


% \mode<presentation>
% {
% \useinnertheme[shadow=true]{rounded}
% \useoutertheme{infolines}
% \usecolortheme{dove}
% \setbeamerfont{block title}{size={}}
% }

\title[Bio-Ontologies]{Semantic similarity}

\author{Robert Hoehndorf}


\date{}

\begin{document}

\begin{frame}
  \titlepage
\end{frame}

\section{Semantic similarity}

\begin{frame}
  Semantic similarity
  \begin{itemize}
  \item We want to use {\em  background knowledge} in ontologies to
    \begin{itemize}
    \item determine similarity between classes,
    \item instances,
    \item and entities with ontology annotations
    \end{itemize}
  \end{itemize}
\end{frame}


\begin{frame}
  \frametitle{How to measure similarity?}
  \begin{itemize}
  \item semantic similarity measures similarity between classes
  \item semantic similarity measures similarity between instances of classes
  \item semantic similarity measures similarity between entities
    {\em associated} with classes
  \item $\Rightarrow$ reduce all of this to similarity between classes
  \end{itemize}
\end{frame}

\begin{frame}
  \frametitle{How to measure similarity?}
  What properties do we want in a similarity measure?
  \\
  A function $sim: D \times D$ is a similarity on $D$ if, for
  all $x, y \in D$, the function $sim$ is:  \begin{itemize}
    \pause
  \item non-negative: $sim(x,y) \geq 0$ for all $x, y$
    \pause
  \item symmetric: $sim(x,y) = sim(y,x)$
    \pause
  \item reflexive: $sim(x,x) = max_D$
    \pause
    \begin{itemize}
    \item weaker form: $sim(x,x) > sim(x,y)$ for all $x \not= y$
    \end{itemize}
    \pause
  \item $sim(x,x) > sim(x,y)$ for $x\not= y$
    \pause
  \item $sim$ is a {\em normalized} similarity measure if it has
    values in $[0,1]$
  \end{itemize}
\end{frame}

\usetikzlibrary{arrows,positioning,automata}
\begin{frame}
  \frametitle{How to measure similarity?}
  \begin{columns}
    \begin{column}{.6\textwidth}
      {\tiny
        \begin{tikzpicture}[>=stealth',shorten >=1pt,node distance=2cm,on grid,initial/.style    ={}]
          \node[state]          (A)                        {$Thing$};
          \node[state]          (B) [below left =of A]    {$Color$};
          \node[state]          (C) [below right =of A]    {$Shape$};
          \node[state]          (D) [below left =of B]    {$Red$};
          \node[state]          (H) [below right =of B]    {$Green$};
          \node[state]          (E) [below  =of D]    {$Orange$};
          \node[state]          (F) [below =of C]    {$Round$};
          \node[state]          (G) [below right =of C]    {$Square$};
          \tikzset{mystyle/.style={->,double=orange}} 
          \tikzset{every node/.style={fill=white}} 
          \path (B)     edge [mystyle]    node   {$isa$} (A)
          (C)     edge [mystyle]    node   {$isa$} (A) 
          (D)     edge [mystyle]    node   {$isa$} (B)
          (H)     edge [mystyle]    node   {$isa$} (B)
          (E)     edge [mystyle]    node   {$isa$} (D)
          (F)     edge [mystyle]    node   {$isa$} (C)
          (G)     edge [mystyle]    node   {$isa$} (C);
          \tikzset{mystyle/.style={<->,double=orange}}   
          \tikzset{mystyle/.style={<->,relative=true,in=0,out=60,double=orange}}
        \end{tikzpicture}
      }
    \end{column}
    \begin{column}{.4\textwidth}
      \begin{itemize}
        \pause
      \item distance on shortest path (Rada {\em et al.}, 1989)
        \pause
      \item $dist_{Rada}(u,v) = sp(u, isa, v)$
        \pause
      \item $sim_{Rada}(u,v) = \frac{1}{dist_{Rada}(u,v) + 1}$
      \end{itemize}
    \end{column}
  \end{columns}
\end{frame}

\begin{frame}
  \frametitle{How to measure similarity?}
  \begin{columns}
    \begin{column}{.6\textwidth}
      {\tiny
        \begin{tikzpicture}[>=stealth',shorten >=1pt,node distance=2cm,on grid,initial/.style    ={}]
          \node[state]          (A)                        {$Thing$};
          \node[state]          (B) [below left =of A]    {$Color$};
          \node[state]          (C) [below right =of A]    {$Shape$};
          \node[state, fill=green]          (D) [below left =of B]    {$Red$};
          \node[state, fill=green]          (H) [below right =of B]    {$Green$};
          \node[state]          (E) [below  =of D]    {$Orange$};
          \node[state]          (F) [below =of C]    {$Round$};
          \node[state]          (G) [below right =of C]    {$Square$};
          \tikzset{mystyle/.style={->,double=orange}} 
          \tikzset{highlight/.style={->,double=green}} 
          \tikzset{every node/.style={fill=white}}
          \path (B)     edge [mystyle]    node   {$isa$} (A)
          (C)     edge [mystyle]    node   {$isa$} (A) 
          (D)     edge [highlight]    node   {$isa$} (B)
          (H)     edge [highlight]    node   {$isa$} (B)
          (E)     edge [mystyle]    node   {$isa$} (D)
          (F)     edge [mystyle]    node   {$isa$} (C)
          (G)     edge [mystyle]    node   {$isa$} (C);
          \tikzset{mystyle/.style={<->,double=orange}}   
          \tikzset{mystyle/.style={<->,relative=true,in=0,out=60,double=orange}}
        \end{tikzpicture}
      }
    \end{column}
    \begin{column}{.4\textwidth}
      \begin{itemize}
      \item distance on shortest path
        \pause
       \item distance(green, red) = 2
       \item $sim_{Rada}(green, red) = \frac{1}{3}$
      \end{itemize}
    \end{column}
  \end{columns}
\end{frame}

\begin{frame}
  \frametitle{How to measure similarity?}
  \begin{columns}
    \begin{column}{.6\textwidth}
      {\tiny
        \begin{tikzpicture}[>=stealth',shorten >=1pt,node distance=2cm,on grid,initial/.style    ={}]
          \node[state]          (A)                        {$Thing$};
          \node[state]          (B) [below left =of A]    {$Color$};
          \node[state]          (C) [below right =of A]    {$Shape$};
          \node[state]          (D) [below left =of B]    {$Red$};
          \node[state]          (H) [below right =of B]    {$Green$};
          \node[state]          (E) [below  =of D]    {$Orange$};
          \node[state, fill=green]          (F) [below =of C]    {$Round$};
          \node[state, fill=green]          (G) [below right =of C]    {$Square$};
          \tikzset{mystyle/.style={->,double=orange}} 
          \tikzset{highlight/.style={->,double=green}} 
          \tikzset{every node/.style={fill=white}}
          \path (B)     edge [mystyle]    node   {$isa$} (A)
          (C)     edge [mystyle]    node   {$isa$} (A) 
          (D)     edge [mystyle]    node   {$isa$} (B)
          (H)     edge [mystyle]    node   {$isa$} (B)
          (E)     edge [mystyle]    node   {$isa$} (D)
          (F)     edge [highlight]    node   {$isa$} (C)
          (G)     edge [highlight]    node   {$isa$} (C);
          \tikzset{mystyle/.style={<->,double=orange}}   
          \tikzset{mystyle/.style={<->,relative=true,in=0,out=60,double=orange}}
        \end{tikzpicture}
      }
    \end{column}
    \begin{column}{.4\textwidth}
      \begin{itemize}
       \item distance on shortest path
       \item distance(square, round) = 2
       \item $sim_{Rada}(square, round) = \frac{1}{3}$
      \end{itemize}
    \end{column}
  \end{columns}
\end{frame}

\begin{frame}
  \frametitle{How to measure similarity?}
  \begin{columns}
    \begin{column}{.6\textwidth}
      {\tiny
        \begin{tikzpicture}[>=stealth',shorten >=1pt,node distance=2cm,on grid,initial/.style    ={}]
          \node[state]          (A)                        {$Thing$};
          \node[state, fill=green]          (B) [below left =of A]    {$Color$};
          \node[state]          (C) [below right =of A]    {$Shape$};
          \node[state]          (D) [below left =of B]    {$Red$};
          \node[state]          (H) [below right =of B]    {$Green$};
          \node[state, fill=green]          (E) [below  =of D]    {$Orange$};
          \node[state]          (F) [below =of C]    {$Round$};
          \node[state]          (G) [below right =of C]    {$Square$};
          \tikzset{mystyle/.style={->,double=orange}} 
          \tikzset{highlight/.style={->,double=green}} 
          \tikzset{every node/.style={fill=white}}
          \path (B)     edge [mystyle]    node   {$isa$} (A)
          (C)     edge [mystyle]    node   {$isa$} (A) 
          (D)     edge [highlight]    node   {$isa$} (B)
          (H)     edge [mystyle]    node   {$isa$} (B)
          (E)     edge [highlight]    node   {$isa$} (D)
          (F)     edge [mystyle]    node   {$isa$} (C)
          (G)     edge [mystyle]    node   {$isa$} (C);
          \tikzset{mystyle/.style={<->,double=orange}}   
          \tikzset{mystyle/.style={<->,relative=true,in=0,out=60,double=orange}}
        \end{tikzpicture}
      }
    \end{column}
    \begin{column}{.4\textwidth}
      \begin{itemize}
       \item distance on shortest path
       \item distance(orange, color) = 2
       \item $sim_{Rada}(orange, color) = \frac{1}{3}$
      \end{itemize}
    \end{column}
  \end{columns}
\end{frame}

\begin{frame}
  \frametitle{How to measure similarity?}
  \begin{itemize}
  \item shortest path is not always intuitive
    \pause
  \item we need a way to determine {\em specificity} of a class
    \begin{itemize}
    \item number of ancestors
    \item number of children
    \item information content
    \end{itemize}
    \pause
  \item {\em density} of a branch in the ontology
    \begin{itemize}
    \item number of siblings
    \item information content
    \end{itemize}
    \pause
  \item account for different edge types
    \begin{itemize}
    \item non-uniform edge weighting
    \end{itemize}
  \end{itemize}
\end{frame}

\begin{frame}
  \frametitle{How to measure similarity}
  \begin{itemize}
  \item term specificity measure $\sigma: C \mapsto \mathbb{R}$:
    \begin{itemize}
    \item $x \sqsubseteq y \rightarrow \sigma(x) \geq \sigma(y)$
    \end{itemize}
    \pause
  \item intrinsic:
    \begin{itemize}
    \item $\sigma(x) = f(depth(x))$
    \item $\sigma(x) = f(A(x))$ (for ancestors $A(x)$)
    \item $\sigma(x) = f(D(x))$ (for descendants $D(x)$)
    \item many more, e.g., Zhou et al.: $\sigma(x) = k \cdot \Big( 1-\frac{\log
        |D(x)|}{\log |C|} \Big) + (1-k) \frac{\log depth(x)}{\log
        depth(G_T)} $
    \end{itemize}
    \pause
  \item extrinsic:
    \begin{itemize}
    \item $\sigma(x)$ defined as a function of instances (or annotations) $I$
      \begin{itemize}
      \item note: the number of instances monotonically decreases with
        increasing depth in taxonomies
      \end{itemize}
    \item Resnik 1995: $eIC_{Resnik}(x) = -\log p(x)$ (with $p(x) =
      \frac{|I(x)|}{|I|}$)
      \begin{itemize}
      \item one of the most popular specificity measure when
        instances are present
      \end{itemize}
    \end{itemize}
  \end{itemize}
\end{frame}

\begin{frame}
  \frametitle{How to measure similarity?}
  \begin{columns}
    \begin{column}{.6\textwidth}
      {\tiny
        \begin{tikzpicture}[>=stealth',shorten >=1pt,node distance=2cm,on grid,initial/.style    ={}]
          \node[state,label=below:$0.0$]          (A)                        {$Thing$};
          \node[state,label=below:$1.0$]          (B) [below left =of A]    {$Color$};
          \node[state,label=right:$1.0$]          (C) [below right =of A]    {$Shape$};
          \node[state,label=right:$2.0$]          (D) [below left =of B]    {$Red$};
          \node[state,label=below:$2.0$]          (H) [below right =of B]    {$Green$};
          \node[state,label=below:$3.0$]          (E) [below  =of D]    {$Orange$};
          \node[state,label=below:$2.0$]          (F) [below =of C]    {$Round$};
          \node[state,label=below:$2.0$]          (G) [below right =of C]    {$Square$};
          \tikzset{mystyle/.style={->,double=orange}} 
          \tikzset{highlight/.style={->,double=green}} 
          \tikzset{every node/.style={fill=white}}
          \path (B)     edge [mystyle]    node   {$isa$} (A)
          (C)     edge [mystyle]    node   {$isa$} (A) 
          (D)     edge [mystyle]    node   {$isa$} (B)
          (H)     edge [mystyle]    node   {$isa$} (B)
          (E)     edge [mystyle]    node   {$isa$} (D)
          (F)     edge [mystyle]    node   {$isa$} (C)
          (G)     edge [mystyle]    node   {$isa$} (C);
          \tikzset{mystyle/.style={<->,double=orange}}   
          \tikzset{mystyle/.style={<->,relative=true,in=0,out=60,double=orange}}
        \end{tikzpicture}
      }
    \end{column}
    \begin{column}{.4\textwidth}
      \begin{itemize}
      \item Resnik 1995: similarity between $x$ and $y$ is the
        information content of the {\em most informative common
          ancestor}
      \end{itemize}
    \end{column}
  \end{columns}
\end{frame}

\begin{frame}
  \frametitle{How to measure similarity?}
  \begin{columns}
    \begin{column}{.6\textwidth}
      {\tiny
        \begin{tikzpicture}[>=stealth',shorten >=1pt,node distance=2cm,on grid,initial/.style    ={}]
          \node[state,label=below:$0.0$]          (A)                        {$Thing$};
          \node[state,label=below:$1.0$]          (B) [below left =of A]    {$Color$};
          \node[state,label=right:$1.0$]          (C) [below right =of A]    {$Shape$};
          \node[state,fill=green,label=right:$2.0$]          (D) [below left =of B]    {$Red$};
          \node[state,fill=green,label=below:$2.0$]          (H) [below right =of B]    {$Green$};
          \node[state,label=below:$3.0$]          (E) [below  =of D]    {$Orange$};
          \node[state,label=below:$2.0$]          (F) [below =of C]    {$Round$};
          \node[state,label=below:$2.0$]          (G) [below right =of C]    {$Square$};
          \tikzset{mystyle/.style={->,double=orange}} 
          \tikzset{highlight/.style={->,double=green}} 
          \tikzset{every node/.style={fill=white}}
          \path (B)     edge [mystyle]    node   {$isa$} (A)
          (C)     edge [mystyle]    node   {$isa$} (A) 
          (D)     edge [mystyle]    node   {$isa$} (B)
          (H)     edge [mystyle]    node   {$isa$} (B)
          (E)     edge [mystyle]    node   {$isa$} (D)
          (F)     edge [mystyle]    node   {$isa$} (C)
          (G)     edge [mystyle]    node   {$isa$} (C);
          \tikzset{mystyle/.style={<->,double=orange}}   
          \tikzset{mystyle/.style={<->,relative=true,in=0,out=60,double=orange}}
        \end{tikzpicture}
      }
    \end{column}
    \begin{column}{.4\textwidth}
      \begin{itemize}
      \item Resnik 1995: similarity between $x$ and $y$ is the
        information content of the {\em most informative common
          ancestor}
      \end{itemize}
    \end{column}
  \end{columns}
\end{frame}

\begin{frame}
  \frametitle{How to measure similarity?}
  \begin{columns}
    \begin{column}{.6\textwidth}
      {\tiny
        \begin{tikzpicture}[>=stealth',shorten >=1pt,node distance=2cm,on grid,initial/.style    ={}]
          \node[state,label=below:$0.0$]          (A)                        {$Thing$};
          \node[state,fill=yellow,label=below:$1.0$]          (B) [below left =of A]    {$Color$};
          \node[state,label=right:$1.0$]          (C) [below right =of A]    {$Shape$};
          \node[state,fill=green,label=right:$2.0$]          (D) [below left =of B]    {$Red$};
          \node[state,fill=green,label=below:$2.0$]          (H) [below right =of B]    {$Green$};
          \node[state,label=below:$3.0$]          (E) [below  =of D]    {$Orange$};
          \node[state,label=below:$2.0$]          (F) [below =of C]    {$Round$};
          \node[state,label=below:$2.0$]          (G) [below right =of C]    {$Square$};
          \tikzset{mystyle/.style={->,double=orange}} 
          \tikzset{highlight/.style={->,double=green}} 
          \tikzset{every node/.style={fill=white}}
          \path (B)     edge [mystyle]    node   {$isa$} (A)
          (C)     edge [mystyle]    node   {$isa$} (A) 
          (D)     edge [mystyle]    node   {$isa$} (B)
          (H)     edge [mystyle]    node   {$isa$} (B)
          (E)     edge [mystyle]    node   {$isa$} (D)
          (F)     edge [mystyle]    node   {$isa$} (C)
          (G)     edge [mystyle]    node   {$isa$} (C);
          \tikzset{mystyle/.style={<->,double=orange}}   
          \tikzset{mystyle/.style={<->,relative=true,in=0,out=60,double=orange}}
        \end{tikzpicture}
      }
    \end{column}
    \begin{column}{.4\textwidth}
      \begin{itemize}
      \item Resnik 1995: similarity between $x$ and $y$ is the
        information content of the {\em most informative common
          ancestor}
      \end{itemize}
    \end{column}
  \end{columns}
\end{frame}

\begin{frame}
  \frametitle{How to measure similarity?}
  \begin{columns}
    \begin{column}{.6\textwidth}
      {\tiny
        \begin{tikzpicture}[>=stealth',shorten >=1pt,node distance=2cm,on grid,initial/.style    ={}]
          \node[state,label=below:$0.0$]          (A)                        {$Thing$};
          \node[state,fill=yellow,label=below:$1.0$]          (B) [below left =of A]    {$Color$};
          \node[state,label=right:$1.0$]          (C) [below right =of A]    {$Shape$};
          \node[state,fill=green,label=right:$2.0$]          (D) [below left =of B]    {$Red$};
          \node[state,fill=green,label=below:$2.0$]          (H) [below right =of B]    {$Green$};
          \node[state,label=below:$3.0$]          (E) [below  =of D]    {$Orange$};
          \node[state,label=below:$2.0$]          (F) [below =of C]    {$Round$};
          \node[state,label=below:$2.0$]          (G) [below right =of C]    {$Square$};
          \tikzset{mystyle/.style={->,double=orange}} 
          \tikzset{highlight/.style={->,double=green}} 
          \tikzset{every node/.style={fill=white}}
          \path (B)     edge [mystyle]    node   {$isa$} (A)
          (C)     edge [mystyle]    node   {$isa$} (A) 
          (D)     edge [mystyle]    node   {$isa$} (B)
          (H)     edge [mystyle]    node   {$isa$} (B)
          (E)     edge [mystyle]    node   {$isa$} (D)
          (F)     edge [mystyle]    node   {$isa$} (C)
          (G)     edge [mystyle]    node   {$isa$} (C);
          \tikzset{mystyle/.style={<->,double=orange}}   
          \tikzset{mystyle/.style={<->,relative=true,in=0,out=60,double=orange}}
        \end{tikzpicture}
      }
    \end{column}
    \begin{column}{.4\textwidth}
      \begin{itemize}
      \item Resnik 1995: similarity between $x$ and $y$ is the
        information content of the {\em most informative common
          ancestor}
        \item $sim_{Resnik}(Green, Red) = 1.0$
      \end{itemize}
    \end{column}
  \end{columns}
\end{frame}

\begin{frame}
  \frametitle{How to measure similarity?}
  \begin{columns}
    \begin{column}{.6\textwidth}
      {\tiny
        \begin{tikzpicture}[>=stealth',shorten >=1pt,node distance=2cm,on grid,initial/.style    ={}]
          \node[state,label=below:$0.0$]          (A)                        {$Thing$};
          \node[state,fill=yellow,label=below:$1.0$]          (B) [below left =of A]    {$Color$};
          \node[state,label=right:$1.0$]          (C) [below right =of A]    {$Shape$};
          \node[state,label=right:$2.0$]          (D) [below left =of B]    {$Red$};
          \node[state,fill=green,label=below:$2.0$]          (H) [below right =of B]    {$Green$};
          \node[state,fill=green,label=below:$3.0$]          (E) [below  =of D]    {$Orange$};
          \node[state,label=below:$2.0$]          (F) [below =of C]    {$Round$};
          \node[state,label=below:$2.0$]          (G) [below right =of C]    {$Square$};
          \tikzset{mystyle/.style={->,double=orange}} 
          \tikzset{highlight/.style={->,double=green}} 
          \tikzset{every node/.style={fill=white}}
          \path (B)     edge [mystyle]    node   {$isa$} (A)
          (C)     edge [mystyle]    node   {$isa$} (A) 
          (D)     edge [mystyle]    node   {$isa$} (B)
          (H)     edge [mystyle]    node   {$isa$} (B)
          (E)     edge [mystyle]    node   {$isa$} (D)
          (F)     edge [mystyle]    node   {$isa$} (C)
          (G)     edge [mystyle]    node   {$isa$} (C);
          \tikzset{mystyle/.style={<->,double=orange}}   
          \tikzset{mystyle/.style={<->,relative=true,in=0,out=60,double=orange}}
        \end{tikzpicture}
      }
    \end{column}
    \begin{column}{.4\textwidth}
      \begin{itemize}
      \item Resnik 1995: similarity between $x$ and $y$ is the
        information content of the {\em most informative common
          ancestor}
        \item $sim_{Resnik}(Green, Orange) = 1.0$
      \end{itemize}
    \end{column}
  \end{columns}
\end{frame}

\begin{frame}
  \frametitle{How to measure similarity?}
  \begin{columns}
    \begin{column}{.6\textwidth}
      {\tiny
        \begin{tikzpicture}[>=stealth',shorten >=1pt,node distance=2cm,on grid,initial/.style    ={}]
          \node[state,fill=yellow,label=below:$0.0$]          (A)                        {$Thing$};
          \node[state,label=below:$1.0$]          (B) [below left =of A]    {$Color$};
          \node[state,label=right:$1.0$]          (C) [below right =of A]    {$Shape$};
          \node[state,label=right:$2.0$]          (D) [below left =of B]    {$Red$};
          \node[state,fill=green,label=below:$2.0$]          (H) [below right =of B]    {$Green$};
          \node[state,label=below:$3.0$]          (E) [below  =of D]    {$Orange$};
          \node[state,label=below:$2.0$]          (F) [below =of C]    {$Round$};
          \node[state,fill=green,label=below:$2.0$]          (G) [below right =of C]    {$Square$};
          \tikzset{mystyle/.style={->,double=orange}} 
          \tikzset{highlight/.style={->,double=green}} 
          \tikzset{every node/.style={fill=white}}
          \path (B)     edge [mystyle]    node   {$isa$} (A)
          (C)     edge [mystyle]    node   {$isa$} (A) 
          (D)     edge [mystyle]    node   {$isa$} (B)
          (H)     edge [mystyle]    node   {$isa$} (B)
          (E)     edge [mystyle]    node   {$isa$} (D)
          (F)     edge [mystyle]    node   {$isa$} (C)
          (G)     edge [mystyle]    node   {$isa$} (C);
          \tikzset{mystyle/.style={<->,double=orange}}   
          \tikzset{mystyle/.style={<->,relative=true,in=0,out=60,double=orange}}
        \end{tikzpicture}
      }
    \end{column}
    \begin{column}{.4\textwidth}
      \begin{itemize}
      \item Resnik 1995: similarity between $x$ and $y$ is the
        information content of the {\em most informative common
          ancestor}
        \item $sim_{Resnik}(Square, Orange) = 0.0$
      \end{itemize}
    \end{column}
  \end{columns}
\end{frame}

\begin{frame}
  \frametitle{How to measure similarity?}
  \begin{itemize}
  \item (Red, Green) and (Orange, Green) have the same similarity
  \item need to incorporate the specificity of the compared classes
  \end{itemize}
\end{frame}

\begin{frame}
  \frametitle{How to measure similarity?}
  \begin{columns}
    \begin{column}{.6\textwidth}
      {\tiny
        \begin{tikzpicture}[>=stealth',shorten >=1pt,node distance=2cm,on grid,initial/.style    ={}]
          \node[state,label=below:$0.0$]          (A)                        {$Thing$};
          \node[state,fill=yellow,label=below:$1.0$]          (B) [below left =of A]    {$Color$};
          \node[state,label=right:$1.0$]          (C) [below right =of A]    {$Shape$};
          \node[state,fill=green,label=right:$2.0$]          (D) [below left =of B]    {$Red$};
          \node[state,fill=green,label=below:$2.0$]          (H) [below right =of B]    {$Green$};
          \node[state,label=below:$3.0$]          (E) [below  =of D]    {$Orange$};
          \node[state,label=below:$2.0$]          (F) [below =of C]    {$Round$};
          \node[state,label=below:$2.0$]          (G) [below right =of C]    {$Square$};
          \tikzset{mystyle/.style={->,double=orange}} 
          \tikzset{highlight/.style={->,double=green}} 
          \tikzset{every node/.style={fill=white}}
          \path (B)     edge [mystyle]    node   {$isa$} (A)
          (C)     edge [mystyle]    node   {$isa$} (A) 
          (D)     edge [mystyle]    node   {$isa$} (B)
          (H)     edge [mystyle]    node   {$isa$} (B)
          (E)     edge [mystyle]    node   {$isa$} (D)
          (F)     edge [mystyle]    node   {$isa$} (C)
          (G)     edge [mystyle]    node   {$isa$} (C);
          \tikzset{mystyle/.style={<->,double=orange}}   
          \tikzset{mystyle/.style={<->,relative=true,in=0,out=60,double=orange}}
        \end{tikzpicture}
      }
    \end{column}
    \begin{column}{.4\textwidth}
      \begin{itemize}
      \item Lin 1998: $sim_{Lin}(x,y) = \frac{2\cdot
          IC(MICA(x,y))}{IC(x) + IC(y)}$
        \pause
      \item $sim_{Lin}(Green, Red) = 0.5$
      \end{itemize}
    \end{column}
  \end{columns}
\end{frame}

\begin{frame}
  \frametitle{How to measure similarity?}
  \begin{columns}
    \begin{column}{.6\textwidth}
      {\tiny
        \begin{tikzpicture}[>=stealth',shorten >=1pt,node distance=2cm,on grid,initial/.style    ={}]
          \node[state,label=below:$0.0$]          (A)                        {$Thing$};
          \node[state,fill=yellow,label=below:$1.0$]          (B) [below left =of A]    {$Color$};
          \node[state,label=right:$1.0$]          (C) [below right =of A]    {$Shape$};
          \node[state,label=right:$2.0$]          (D) [below left =of B]    {$Red$};
          \node[state,fill=green,label=below:$2.0$]          (H) [below right =of B]    {$Green$};
          \node[state,fill=green,label=below:$3.0$]          (E) [below  =of D]    {$Orange$};
          \node[state,label=below:$2.0$]          (F) [below =of C]    {$Round$};
          \node[state,label=below:$2.0$]          (G) [below right =of C]    {$Square$};
          \tikzset{mystyle/.style={->,double=orange}} 
          \tikzset{highlight/.style={->,double=green}} 
          \tikzset{every node/.style={fill=white}}
          \path (B)     edge [mystyle]    node   {$isa$} (A)
          (C)     edge [mystyle]    node   {$isa$} (A) 
          (D)     edge [mystyle]    node   {$isa$} (B)
          (H)     edge [mystyle]    node   {$isa$} (B)
          (E)     edge [mystyle]    node   {$isa$} (D)
          (F)     edge [mystyle]    node   {$isa$} (C)
          (G)     edge [mystyle]    node   {$isa$} (C);
          \tikzset{mystyle/.style={<->,double=orange}}   
          \tikzset{mystyle/.style={<->,relative=true,in=0,out=60,double=orange}}
        \end{tikzpicture}
      }
    \end{column}
    \begin{column}{.4\textwidth}
      \begin{itemize}
      \item Lin 1998: $sim_{Lin}(x,y) = \frac{2\cdot
          IC(MICA(x,y))}{IC(x) + IC(y)}$
      \item $sim_{Lin}(Green, Orange) = 0.4$
      \end{itemize}
    \end{column}
  \end{columns}
\end{frame}

\begin{frame}
  \frametitle{How to measure similarity?}
  \begin{itemize}
  \item many(!) others:
    \begin{itemize}
    \item Jiang \& Conrath 1997
    \item Mazandu \& Mulder 2013
    \item Schlicker et al. 2009
    \item ...
  \end{itemize}
  \end{itemize}
\end{frame}

\begin{frame}
  \frametitle{How to measure similarity?}
  \begin{itemize}
  \item we only looked at comparing pairs of classes
  \item mostly, we want to compare {\em sets} of classes
    \begin{itemize}
    \item set of GO annotations
    \item set of signs and symptoms
    \item set of phenotypes
    \end{itemize}
  \item two approaches:
    \begin{itemize}
    \item compare each class individually, then merge
    \item directly set-based similarity measures
    \end{itemize}
  \end{itemize}
\end{frame}

\begin{frame}
  \frametitle{How to measure similarity?}
  \begin{columns}
    \begin{column}{.6\textwidth}
      {\tiny
        \begin{tikzpicture}[>=stealth',shorten >=1pt,node distance=2cm,on grid,initial/.style    ={}]
          \node[state,label=below:$0.0$]          (A)                        {$Thing$};
          \node[state,label=below:$1.0$]          (B) [below left =of A]    {$Color$};
          \node[state,label=right:$1.0$]          (C) [below right =of A]    {$Shape$};
          \node[state,fill=gray,label=right:$2.0$]          (D) [below left =of B]    {$Red$};
          \node[state,label=below:$2.0$]          (H) [below right =of B]    {$Green$};
          \node[state,fill=green,label=below:$3.0$]          (E) [below  =of D]    {$Orange$};
          \node[state,fill=gray,label=below:$2.0$]          (F) [below =of C]    {$Round$};
          \node[state,fill=green,label=below:$2.0$]          (G) [below right =of C]    {$Square$};
          \tikzset{mystyle/.style={->,double=orange}} 
          \tikzset{highlight/.style={->,double=green}} 
          \tikzset{every node/.style={fill=white}}
          \path (B)     edge [mystyle]    node   {$isa$} (A)
          (C)     edge [mystyle]    node   {$isa$} (A) 
          (D)     edge [mystyle]    node   {$isa$} (B)
          (H)     edge [mystyle]    node   {$isa$} (B)
          (E)     edge [mystyle]    node   {$isa$} (D)
          (F)     edge [mystyle]    node   {$isa$} (C)
          (G)     edge [mystyle]    node   {$isa$} (C);
          \tikzset{mystyle/.style={<->,double=orange}}   
          \tikzset{mystyle/.style={<->,relative=true,in=0,out=60,double=orange}}
        \end{tikzpicture}
      }
    \end{column}
    \begin{column}{.4\textwidth}
      \begin{itemize}
      \item similarity between a square-and-orange thing and a
        round-and-red thing
        \pause
      \item Pesquita et al., 2007:
        $simGIC(X,Y) = \frac{\sum_{c \in A(X) \cap A(Y)}
          IC(c)}{\sum_{c \in A(X) \cup A(Y)} IC(c)}$
      \end{itemize}
    \end{column}
  \end{columns}
\end{frame}

\begin{frame}
  \frametitle{How to measure similarity?}
  \begin{columns}
    \begin{column}{.6\textwidth}
      {\tiny
        \begin{tikzpicture}[>=stealth',shorten >=1pt,node distance=2cm,on grid,initial/.style    ={}]
          \node[state,fill=pink,label=below:$0.0$]          (A)                        {$Thing$};
          \node[state,fill=pink,label=below:$1.0$]          (B) [below left =of A]    {$Color$};
          \node[state,fill=pink,label=right:$1.0$]          (C) [below right =of A]    {$Shape$};
          \node[state,fill=gray,label=right:$2.0$]          (D) [below left =of B]    {$Red$};
          \node[state,label=below:$2.0$]          (H) [below right =of B]    {$Green$};
          \node[state,fill=green,label=below:$3.0$]          (E) [below  =of D]    {$Orange$};
          \node[state,fill=gray,label=below:$2.0$]          (F) [below =of C]    {$Round$};
          \node[state,fill=green,label=below:$2.0$]          (G) [below right =of C]    {$Square$};
          \tikzset{mystyle/.style={->,double=orange}} 
          \tikzset{highlight/.style={->,double=green}} 
          \tikzset{every node/.style={fill=white}}
          \path (B)     edge [mystyle]    node   {$isa$} (A)
          (C)     edge [mystyle]    node   {$isa$} (A) 
          (D)     edge [mystyle]    node   {$isa$} (B)
          (H)     edge [mystyle]    node   {$isa$} (B)
          (E)     edge [mystyle]    node   {$isa$} (D)
          (F)     edge [mystyle]    node   {$isa$} (C)
          (G)     edge [mystyle]    node   {$isa$} (C);
          \tikzset{mystyle/.style={<->,double=orange}}   
          \tikzset{mystyle/.style={<->,relative=true,in=0,out=60,double=orange}}
        \end{tikzpicture}
      }
    \end{column}
    \begin{column}{.4\textwidth}
      \begin{itemize}
      \item similarity between a square-and-orange thing and a
        round-and-red thing
      \item Pesquita et al., 2007:
        $simGIC(X,Y) = \frac{\sum_{c \in A(X) \cap A(Y)}
          IC(c)}{\sum_{c \in A(X) \cup A(Y)} IC(c)}$
      \item $simGIC(so,rr) = \frac{2}{11}$
      \end{itemize}
    \end{column}
  \end{columns}
\end{frame}

\begin{frame}
  \frametitle{How to measure similarity?}
  \begin{itemize}
  \item alternatively: use different merging strategies
  \item common: average, maximum, {\bf best-matching average}
    \begin{itemize}
    \item Average: $sim_A(X,Y) = \frac{\sum_{x\in X} \sum_{y \in Y} sim(x,y)}{|X| \times |Y|}$
    \item Max average: $sim_{MA}(X,Y) = \frac{1}{|X|} \sum_{x\in X} \max_{y \in Y} sim(x,y)$
    \item Best match average: $sim_{BMA}(X,Y) = \frac{sim_{MA}(X,Y) + sim_{MA}(Y,X)}{2}$
    \end{itemize}
  \end{itemize}
\end{frame}

\begin{frame}
  \frametitle{How to measure similarity?}
  \begin{itemize}
  \item Semantic Measures Library:
    \begin{itemize}
    \item comprehensive Java library
    \item \url{http://www.semantic-measures-library.org/}
    \end{itemize}
  \item R packages: GOSim, GOSemSim, HPOSim, LSAfun,
    ontologySimilarity,...
  \item Python: sematch, fastsemsim (GO only)
  \end{itemize}
\end{frame}

\begin{frame}
  \frametitle{Applications of semantic similarity}
  \begin{itemize}
  \item no obvious choice of similarity measure
  \item depends on application
    % \begin{itemize}
    % \item e.g., predicting PPIs in different organisms through
    %   similarity may benefit from a different similarity measure!
    % \end{itemize}
  \item different similarity measures may react differently to biases
    in data
  \item needs some testing and experience
  \end{itemize}
\end{frame}

\begin{frame}
  \frametitle{Applications of semantic similarity}
  Recommendations:
  \begin{itemize}
  \item use Resnik's information content measure (normalized)
  \item use Resnik's similarity
  \item use Best Match Average
  \item use the full ontology
  \item classify your ontology using an automated reasoner before
    applying semantic similarity
    % \begin{itemize}
    % \item although many ontologies come pre-classified
    % \end{itemize}
  \item $\Rightarrow$ but there are many exceptions
    % \begin{itemize}
    % \item similar location $\Rightarrow$ use location subset of GO
    % \item developmental phenotypes $\Rightarrow$ use developmental
    %   branch of phenotype ontology
    % \end{itemize}
  \end{itemize}
\end{frame}

\begin{frame}
  \frametitle{Use cases}
  Hypothesis: proteins with similar functions are more likely to interact
  \begin{itemize}
  \item ontology: Gene Ontology
  \item no instances but ``annotations''
    \begin{itemize}
    \item protein $P$ associated with classes $C_{P_1},...,C_{P_n}$
    \end{itemize}
  \item formulate as ranking problem
    \begin{itemize}
    \item given a set of proteins $P$, for each protein $p \in P$,
      return a sorted list $l_p$ of protein--score pairs such that
      $l_p$ contains pairs with all proteins in $P$, and each protein
      in $P$ occurs exactly once in a pair in $l_p$
    \end{itemize}
  \item higher-ranked proteins in $l_p$ are more likely to interact
    with $p$
  \item easy to find a similar formulation for other ranking problems
    \begin{itemize}
    \item document retrieval, etc.
    \end{itemize}

  \end{itemize}
\end{frame}

\end{document}
%%% Local Variables:
%%% mode: latex
%%% TeX-master: t
%%% End:
